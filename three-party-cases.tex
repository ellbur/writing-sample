% Options for packages loaded elsewhere
\PassOptionsToPackage{unicode}{hyperref}
\PassOptionsToPackage{hyphens}{url}
%
\documentclass[
  12pt,
  letterpaper,
]{scrartcl}

\usepackage[
  margin=1in,
  footskip=0.5in,
  footnotesep=14pt
]{geometry}

\usepackage{setspace}

\usepackage{enumitem}% http://ctan.org/pkg/enumitem
\setlist[itemize]{leftmargin=0.6in, itemsep=0.25em, topsep=0.25em, first=\setstretch{1.0}}
\setlist[enumerate]{leftmargin=0.6in, itemsep=0.25em, topsep=0.5em, first=\setstretch{1.0}}

\setlength{\parindent}{0.5in}
\setlength{\skip\footins}{10mm plus 2mm}
\setlength{\emergencystretch}{3em} % prevent overfull lines

\renewenvironment{quote}%
  {\list{}{\leftmargin=0.6in\rightmargin=0.6in}\item[]\setstretch{1.0}}%
  {\endlist\vspace{-\topsep}}

\usepackage{amsmath,amssymb}
\usepackage{mathptmx}
\usepackage[T1]{fontenc}
\usepackage[utf8]{inputenc}
\usepackage{textcomp}
\usepackage{xcolor}
\IfFileExists{xurl.sty}{\usepackage{xurl}}{} % add URL line breaks if available

\usepackage{hyperref}
\hypersetup{
  pdftitle={Three-Party Windfall Cases},
  hidelinks,
  pdfcreator={LaTeX via pandoc}}

\urlstyle{same} % disable monospaced font for URLs
\usepackage{longtable,booktabs,array}
\usepackage{multirow}
\usepackage{calc} % for calculating minipage widths
\usepackage{graphicx}

\makeatletter
\def\maxwidth{\ifdim\Gin@nat@width>\linewidth\linewidth\else\Gin@nat@width\fi}
\def\maxheight{\ifdim\Gin@nat@height>\textheight\textheight\else\Gin@nat@height\fi}
\makeatother

\setkeys{Gin}{width=\maxwidth,height=\maxheight,keepaspectratio}

\providecommand{\tightlist}{}

\usepackage{newunicodechar}
\newunicodechar{✔}{\checkmark}

\usepackage[xspace]{ellipsis}

\usepackage{soulutf8}
\setuldepth{o}
\renewcommand{\footnotesize}{\normalsize}

\addtokomafont{disposition}{\rmfamily}
\addtokomafont{section}{\normalsize}
\addtokomafont{subsection}{\normalsize}
\addtokomafont{subsubsection}{\normalsize}
\renewcommand{\thesection}{\Roman{section}} 
\renewcommand{\thesubsection}{\Alph{subsection}}
\renewcommand{\thesubsubsection}{\arabic{subsubsection}}

\renewcommand*{\sectionformat}{\thesection.\enskip}
\renewcommand*{\subsectionformat}{\makebox[0.5in]{}\thesubsection.\enskip}
\renewcommand*{\subsubsectionformat}{\makebox[1in]{}\thesubsubsection.\enskip}

\RedeclareSectionCommand[runin=false, afterindent=true,beforeskip=0.5\baselineskip,afterskip=0.5em]{section}
\RedeclareSectionCommand[runin=false, afterindent=true,beforeskip=0.25\baselineskip,afterskip=0in]{subsection}
\RedeclareSectionCommand[runin=false, afterindent=true,beforeskip=0.125\baselineskip,afterskip=0in]{subsubsection}

\makeatletter
\renewcommand{\sectionlinesformat}[4]{%
\ifstr{#1}{section}{%
    \parbox[t]{\linewidth}{%
      \raggedsection\@hangfrom{\hskip #2#3}{#4}\par%
      \vspace{-0.5em}%
      \kern-.75\ht\strutbox\rule{\linewidth}{.8pt}%
    }%
  }{%
    \@hangfrom{\hskip #2#3}{#4}}% 
}
\makeatother


\makeatletter
\renewcommand{\@makefntext}[1]{%
  \setlength{\parindent}{0pt}%
  \begin{list}{}{\setlength{\labelwidth}{6mm}% 1.5em <==================
    \setlength{\leftmargin}{\labelwidth}%
    \setlength{\labelsep}{3pt}%
    \setlength{\itemsep}{0pt}%
    \setlength{\parsep}{0pt}%
    \setlength{\topsep}{0pt}%
    \footnotesize}%
  \item[\@thefnmark\hfil]#1% @makefnmark
  \end{list}%
}
\makeatother

\title{Three-Party Windfall Cases}
\author{}
\date{}

\begin{document}

\setstretch{2.0}

\begin{center}
{\noindent Owen Healy}

{\noindent Writing Sample}
\end{center}

\hypertarget{introduction}{%
\section{Introduction}\label{introduction}}

This memo considers the rights of a contracting party with respect to
third parties on whom it indirectly relies. Delaware cases have not
produced uniform rules applicable to this situation. The existence of
third-party claims may affect the ability of multiple parties
interacting on a single project to allocate risks ahead of time through
their respective contracts. \emph{See} Steven M. Henderson,
\emph{Walking the Line between Contract and Tort in Construction
Disputes: Assessing the Use of Negligent Misrepresentation to Recover
Economic Loss after} Presnell, 95 Ky.~L.J. 145, 165--68 (2006)
(discussing similar issues under Kentucky law).

\hypertarget{facts}{%
\section{Facts}\label{facts}}

Performer contracts with Venue to rent space for a show. Venue
separately contracts with Electrician. Performer's contract with Venue
contains a limitation-on-liability clause limiting Venue's liability to
return of rent paid; the Venue--Electrician contract contains no
limitation. Although Electrician certifies the wiring as compliant, a
fault is later discovered, forcing Performer to cancel the show and lose
sales in excess of rent payments.

What are Performer's rights against Electrician?

\hypertarget{discussion}{%
\section{Discussion}\label{discussion}}

Performer's most likely claim is for negligent misrepresentation. The
discussion below considers whether the elements of negligent
misrepresentation are satisfied, whether the ``economic loss doctrine''
precludes the claim, and whether the limitation on liability in the
Performer--Venue contract affects Performer's rights against
Electrician.

\hypertarget{negligent-misrepresentation}{%
\subsection{Negligent
Misrepresentation}\label{negligent-misrepresentation}}

To make out a claim for negligent misrepresentation, a plaintiff must
establish that ``(1) the defendant had a pecuniary duty to provide
accurate information, (2) the defendant supplied false information, (3)
the defendant failed to exercise reasonable care in obtaining or
communicating the information, and (4) the plaintiff suffered a
pecuniary loss caused by justifiable reliance upon the false
information.'' \emph{PR Acquisitions, LLC v.~Midland Funding LLC}, C.A.
No.~2017-0465-TMR, 2018 WL 2041521, at *13 (Del.~Ch.~Apr.~30, 2018).

Falsity and lack of due care are satisfied by Electrician's conduct.
Duty reliance are less clear. The indirect relationship between
Performer and Electrician complicates the analysis. While it is clear
Electrician owed a pecuniary duty to Venue, determining whether that
duty extends to Performer presents a closer question.

The doctrine of privity does not supply the answer. While ``{[}t{]}he
Delaware case law is divided,'' authority tends toward the view that
privity is not an ``indispensable prerequisite'' in negligent
misrepresentation cases. \emph{See} \emph{Guardian Construction Co.~v.~Tetra Tech Richardson, Inc.},
583 A.2d 1378, 1384, 1386 (Del.~Super.~Ct.~1990) (surveying cases). In \emph{Guardian Construction},
the defendant design engineer supplied inaccurate plans to the
non-party client, who in turn supplied them to the plaintiff
construction company for preparation of a bid. \emph{Id.}~at 1380--81.
The construction company sued the design engineer, claiming the
inaccuracy caused it to underbid and lose money on the job. \emph{Id.}~at 1381. The Superior Court reasoned that because the design engineer
``knew and intended that {[}the plans{]} would be specifically supplied
to and relied upon by project bidders,'' it ``owed a legal duty to
{[}the construction company{]}, as {[}a{]} known and intended
member{[}{]} of this limited class, to supply correct information.''
\emph{Id.} at 1386.

Following \emph{Guardian Construction}, indirect negligence claims have
been allowed by a property owner against a subcontractor, \emph{Arroyo
v. Regal Builders, LLC}, C.A. No.~K13C-12-028-RBY, 2016 WL 5210880, at
*2 (Del.~Super.~Ct.~Sept.~20, 2016), and by a seller against an
accountant that prepared financial statements bearing on the purchaser's
ability to pay, \emph{Carello v. PricewaterhouseCoopers LLP}, C.A.
No.~01C-10-219-RRC, 2002 WL 1454111, at *4--6 (Del.~Super.~Ct.~July 3,
2002). But a federal district court declined to apply \emph{Guardian
Construction} to a nearly identical construction case, reasoning that
\emph{Guardian Construction}`s non-binding holding' had been ``refined
and narrowed'' in the years since. \emph{Kuhn Construction Co.~v. Ocean
\& Coastal Consultants, Inc.}, 844 F. Supp.~2d 519, 529 (D.~Del.~2012).

The lack of an ``indispensable prerequisite'' for privity does not imply
that a duty is owed to all downstream parties anytime it is known the
information may be relied upon. The law creates a duty only when ``such
a relationship exists between the parties that the community will impose
a legal obligation upon one for the benefit of the other.'' \emph{Naidu
v.~Laird}, 539 A.2d 1064, 1070 (Del.~1988) (quoting W.~Keeton, D.~Dobbs,
R.~Keeton, D.~Owen, \emph{Prosser \& Keeton on Torts} §~37, at 236 (5th
ed.~1984)). Electrician's inspection could be viewed as having only been
performed only to assist Venue in satisfying its own contractual
obligations to Performer. The Performer--Venue and Venue--Electrician
contracts support this view, as Performer and Electrician allocated the
risk of defects among themselves with no consideration for third
parties. Accepting this view, Performer and Electrician do not have
``such a relationship'' that a duty will run from Electrician to
Performer.

Reliance faces similar hurdles. While Performer would not have booked
the venue had it known of electrical problems, Venue's own assurances
and reputation likely played a role, and Electrician has no control over
these factors. A related issue is whether, if the quality of the
physical premises is within the scope of the Performer--Venue contract,
it would be justifiable for Performer to expect Electrician to uphold a
different standard of care based on tort law.

Finally, this is not a situation where Performer had no opportunity to
contract with Electrician. A tort duty running from Electrician to
Performer is unnecessary to protect Performer's interests \emph{ex
ante}. It is only after Performer has agreed to a
limitation on liability and suffered a loss that Performer wishes it
had negotiated protections it never sought. ``{[}I{]}t is not for {[}the
court{]} to rewrite the parties' contract to change the allocation of a
set of risks that the parties left unaddressed.'' \emph{Pavik v. George
\& Lynch, Inc.}, 183 A.3d 1258, 1271 (Del.~2018).

In summary, while the state of the law is uncertain, the better view is
that Performer has no negligent misrepresentation claim against
Electrician in these circumstances.

\hypertarget{economic-loss-doctrine}{%
\subsection{Economic Loss Doctrine}\label{economic-loss-doctrine}}

Assuming the elements of negligent misrepresentation are otherwise
satisfied, the question becomes whether the claim is nevertheless
precluded by the ``economic loss doctrine.''

The economic loss doctrine ``prevents a plaintiff from recovering in
tort for losses that are solely economic in nature.'' \emph{Commonwealth
Construction Co.~v. Endecon, Inc.}, No.~C.A. No.~08C-01-266-RRC, 2009 WL
609426, at *4 (Del.~Super.~Ct.~Mar.~9, 2009) (quotation marks
omitted). But the doctrine is subject to certain exceptions, such that
tort claims for pure economic loss are not categorically forbidden. \emph{See}
\emph{id.}

Authority is inconsistent as to whether the economic loss doctrine is an
exception to an otherwise valid tort claim or a factor to be considered
in deciding whether to impose a tort duty in the first place.
\emph{Compare} \emph{Guardian Construction}, 583 A.2d at 1381
(discussing ``whether \ldots{} {[}the plaintiffs{]} may recover purely
economic losses under their negligence claims \ldots{} absent privity of
contract'' and concluding they could because the defendant ``owed a legal duty''), \emph{with} \emph{Marcucilli v. Boardwalk Builders, Inc.},
C.A. No.~99C-02-007, 1999 WL 1568612, at *5 (Del.~Super.~Ct.~Dec.~22,
1999) (``The economic loss doctrine did nothing to alter the duties that
{[}the defendant{]} owed to others{[};{]} it merely provided an
exception to normal negligence principals of liability.''). Thus the law
is unclear as to whether, once it has been concluded that a tort duty
exists, a separate analysis is required to determine whether the claim
is nonetheless barred by the economic loss doctrine.

On the present facts, the issues of duty and economic loss present the
same question: both ask whether the relationship between Performer and
Electrician is such that Electrician should have to answer to Performer
for its defective work. Accordingly, the analysis of the preceding
section suggests that a tort claim should not exist, whether or not this
is viewed as an application of the economic loss doctrine.

\hypertarget{effect-of-the-limitation-on-liability}{%
\subsection{Effect of the Limitation on
Liability}\label{effect-of-the-limitation-on-liability}}

The final issue is whether Electrician may invoke the limitation on
liability in the Performer--Venue contract.

``{[}T{]}o qualify as a third party beneficiary of a contract, (a) the
contracting parties must have intended that the third party beneficiary
benefit from the contract, (b) the benefit must have been intended as a
gift or in satisfaction of a pre-existing obligation to that person, and
(c) the intent to benefit the third party must be a material part of the
parties' purpose in entering into the contract.'' \emph{E.I. DuPont de
Nemours \& Co.~v. Rhone Poulenc Fiber \& Resin Intermediates, S.A.S.},
269 F.3d 187, 196 (3d Cir.~2001). Because the limitation on liability by
its terms applies only to suits by Performer against Venue, the contract
does not evidence an intent to benefit Electrician. For that reason,
Electrician cannot invoke the provision as a defense to Performer's
negligent misrepresentation claim.

In \emph{Rob-Win, Inc.~v.~Lydia Sec.~Monitoring, Inc.}, C.A.
No.~04C-11-276-CLS, 2007 WL 3360036, at *4 (Del.~Super.~Ct.~Apr.~30, 2007), the court dealt with two related but
different situations. There, the contract between the plaintiff client
and non-party intermediary contained a clause expressly applying the
contract's terms to subcontractors; therefore, the client's tort claim
against the subcontractor was subject to limitations in the contract.
\emph{Id.}~at *3. The present facts are distinguished by an absence of such a
term in the Performer--Venue contract. The second situation addressed in
\emph{Rob-Win} was that the subcontract \emph{also} contained a
limitation on liability. The subcontractor could not raise that
limitation as a defense because the client was not a party to it.
\emph{Id.} at *4-5. That reasoning is also inapplicable here, as
Performer is a party to the Performer--Venue contract.

It should be noted that both applying and not applying the limitation on
liability would work a windfall to one party. If Performer's damages are
not limited, Performer avoids the limitation it negotiated by the
fortuity that the fault can be traced to a third party rather than one
of Venue's employees. If Electrician prevails, it receives a protection
it neither negotiated nor paid for. The existence of either reinforces
the view, expressed earlier, that the tort of negligent
misrepresentation should not extend to such indirect suits.

\hypertarget{conclusion}{%
\section{Conclusion}\label{conclusion}}

The law is unclear, but Performer likely cannot satisfy the duty and
reliance elements of a claim for negligent misrepresentation. To the
extent those elements may be satisfied, the economic loss rule and
contractual limitation on liability are inapplicable.

\end{document}
