% Options for packages loaded elsewhere
% vim: wrap
% vim: linebreak
% vim: spell
\documentclass[
  12pt,
  letterpaper,
]{scrartcl}

\usepackage[
  margin=1in,
  footskip=0.5in,
  footnotesep=14pt
]{geometry}

\usepackage{setspace}

\setlength{\parindent}{0.5in}
\setlength{\skip\footins}{10mm plus 2mm}
\setlength{\emergencystretch}{3em} % prevent overfull lines

\renewenvironment{quote}%
  {\list{}{\leftmargin=0.6in\rightmargin=0.6in}\item[]\setstretch{1.0}}%
  {\endlist\vspace{-\topsep}}

\usepackage[T1]{fontenc}
\usepackage[utf8]{inputenc}
\usepackage{textcomp}

\usepackage{calc} % for calculating minipage widths

\usepackage[xspace]{ellipsis}

\usepackage{soulutf8}
\setuldepth{o}
\renewcommand{\footnotesize}{\normalsize}

\addtokomafont{disposition}{\rmfamily}
\addtokomafont{section}{\normalsize\scshape}
\addtokomafont{subsection}{\normalsize}
\addtokomafont{subsubsection}{\normalsize}
\renewcommand{\thesection}{\Roman{section}} 
\renewcommand{\thesubsection}{\Alph{subsection}}
\renewcommand{\thesubsubsection}{\arabic{subsubsection}}

\renewcommand*{\sectionformat}{\thesection.\enskip}
\renewcommand*{\subsectionformat}{\makebox[0.5in]{}\thesubsection.\enskip}
\renewcommand*{\subsubsectionformat}{\makebox[1in]{}\thesubsubsection.\enskip}

\RedeclareSectionCommand[runin=false, afterindent=true,beforeskip=0.5\baselineskip,afterskip=0in]{section}
\RedeclareSectionCommand[runin=false, afterindent=true,beforeskip=0.25\baselineskip,afterskip=0in]{subsection}
\RedeclareSectionCommand[runin=false, afterindent=true,beforeskip=0.125\baselineskip,afterskip=0in]{subsubsection}

\title{Self-Conscious Common Law}
\author{}
\date{}

\begin{document}

\begin{center}
{\noindent Owen Healy}

\vspace{1.0ex}{\noindent \textit{Self-Conscious Common Law}}
\end{center}

\setstretch{2.0}

\vspace{-5.0ex}%
\section{Introduction}

What happens when the law must answer questions about itself? That is---was law
X \emph{clearly established}? Did lawyer Y \emph{misrepresent} law X? Would
court Z have \emph{accepted} law X had the issue been raised? Statements about
statements (``higher-order statements'') have long bothered logicians,
\textit{see} W.V.~Quine, \textit{Philosophy of Logic} 66--68 (2d ed.~1986), so
it is not surprising that they would make trouble for judges as well. But
law-about-law is not just a theoretical curiosity. Untangling it matters, for
example, to judges who must decide whether to submit such ``legal facts'' to
juries. And, as shown below, the way law ``talks about law'' can affect the
substance of legal rights.

This essay surveys how courts have dealt with law-about-law in three discrete
and seemingly unrelated areas. The purpose is to demonstrate that all of these
issues are actually the same---namely, how law talks about itself.

\section{Discussion}

\subsection{Legal Malpractice}

In a legal malpractice case arising out of litigation, the former client
accuses the lawyer of having litigated the case incorrectly. To prove
causation, the plaintiff must show that, had the lawyer acted correctly, the
outcome would have been more favorable. ???. The causation element is
law-about-law: how would another court have behaved differently had it been
presented with the correct litigation strategy? ???

This law-about-law question creates several issues. The first is whether a
judge or jury should decide it. ???. If the question is how a prior jury would
have acted on different evidence, there is general agreement to send that
question to a jury also. ???. But when the question is how a prior \emph{judge}
would have ruled on a different argument, the authority is reversed, and courts
tend to decide those questions for themselves. ???. But the reasoning of those
cases is not convincing. It is doubtful that how a prior court would have
resolved a hypothetical dispute really falls on the ``law'' side of the
law/fact dichotomy. It has none of the hallmarks of law: there are no statutes
addressing it, ??? (???search for source defining ``law''???) The argument that
judges are better than juries at predicting how other judges would act, even if
true, is not a reason for withholding a factual question from the jury. ???.
Juries routinely resolve difficult factual questions. The Seventh Amendment
does not have a footnote for ``except if the judge things they can do a better
job.'' ???. Judges should not be shy of the fact that they, too, are subject to
the laws of cause and effect, and juries can reason about their actions.

To assist in deciding how a prior litigation would have progressed had the
lawyer taken a different action, courts sometimes employee a device called the
``case within a case,'' in which the malpractice plaintiff litigates the
original case to a new judgment and the difference in outcomes is taken to
represent the plaintiff's damages. ???. But it is important to understand that
the case-within-a-case is not actually how the prior litigation would have
resolved---it is an approximation device. Unfortunately, failure to appreciate
this distinction has led to some bad law. ??? Pennsylvania example ???

\subsection{Insurer-Insured}



\subsection{\textit{FTC v.~Actavis}}


\section{Conclusion}


\end{document}
