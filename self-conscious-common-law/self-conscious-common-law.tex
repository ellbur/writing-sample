% Options for packages loaded elsewhere
% vim: wrap
% vim: linebreak
% vim: spell
\documentclass[
  12pt,
  letterpaper,
]{scrartcl}

\usepackage[
  margin=1in,
  footskip=0.5in,
  footnotesep=14pt
]{geometry}

\usepackage{setspace}
\usepackage{array}

\setlength{\parindent}{0.5in}
\setlength{\skip\footins}{10mm plus 2mm}
\setlength{\emergencystretch}{3em} % prevent overfull lines

\renewenvironment{quote}%
  {\list{}{\leftmargin=0.6in\rightmargin=0.6in}\item[]\setstretch{1.0}}%
  {\endlist\vspace{-\topsep}}

\usepackage[T1]{fontenc}
\usepackage[utf8]{inputenc}
\usepackage{textcomp}

\usepackage{calc} % for calculating minipage widths

\usepackage[xspace]{ellipsis}

\usepackage{soulutf8}
\setuldepth{o}
\renewcommand{\footnotesize}{\normalsize}

\addtokomafont{disposition}{\rmfamily}
\addtokomafont{section}{\normalsize\scshape}
\addtokomafont{subsection}{\normalsize}
\addtokomafont{subsubsection}{\normalsize}
\renewcommand{\thesection}{\Roman{section}} 
\renewcommand{\thesubsection}{\Alph{subsection}}
\renewcommand{\thesubsubsection}{\arabic{subsubsection}}

\renewcommand*{\sectionformat}{\thesection.\enskip}
\renewcommand*{\subsectionformat}{\makebox[0.5in]{}\thesubsection.\enskip}
\renewcommand*{\subsubsectionformat}{\makebox[1in]{}\thesubsubsection.\enskip}

\RedeclareSectionCommand[runin=false, afterindent=true,beforeskip=0.5\baselineskip,afterskip=0in]{section}
\RedeclareSectionCommand[runin=false, afterindent=true,beforeskip=0.25\baselineskip,afterskip=0in]{subsection}
\RedeclareSectionCommand[runin=false, afterindent=true,beforeskip=0.125\baselineskip,afterskip=0in]{subsubsection}

\title{Self-Conscious Law}
\author{Owen E. Healy}

\begin{document}

\begin{center}\vspace{1.0ex}{\noindent \large\textit{Self-Conscious Law}}\end{center}
  
\vspace{12pt}\noindent\begin{minipage}{4in}
  \setlength{\parskip}{6pt}
  
  \noindent By: Owen E. Healy

  \noindent Written as an application writing sample. Not edited by any other person (or robot).
\end{minipage}

\setstretch{2.0}

\vspace{12pt}\section{Introduction}

Statements about statements (``second-order statements'') have long bothered
logicians, \textit{see} W.V.~Quine, \textit{Philosophy of Logic} 66--68 (2d
ed.~1986), so it is not surprising that they would make trouble for judges as
well. But law-about-law is not just a theoretical curiosity. Everyday examples
include: Was law X \emph{clearly established}? Did a lawyer \emph{misrepresent}
law X? Would a court have \emph{accepted} law X had the issue been raised?
Untangling these questions matters to judges who must decide whether to submit
such ``legal facts'' to juries. And, as shown below, the way law ``talks about
law'' can affect the substance of legal rights.

This essay surveys how courts deal with law-about-law in three discrete and
seemingly unrelated areas. The purpose is to demonstrate that all three are
variations of the same underlying theme: how law talks about itself.


\section{Discussion}

Three examples of law-about-law are discussed below: legal malpractice,
liability insurance, and the ``reverse-payment'' antitrust claim recognized
in \textit{FTC v.~Actavis}, 570 U.S.~136 (2013).


\subsection{Legal Malpractice}

In a legal malpractice case arising out of litigation, the former client
accuses the lawyer of having litigated the case incorrectly. To prove
causation, the plaintiff must show that under the correct litigation strategy the
outcome would have been more favorable. \textit{Bozelko v.~Papastavros}, 147
A.3d 1023, 1029 (Conn.~2016). The causation element is therefore law-about-law:
how would another court have behaved if presented with a different approach? \textit{See} \textit{id.}

This law-about-law element creates several issues. The first is whether a judge
or jury should decide it. If the question is how a prior jury would have acted
on different evidence, there is general agreement that a jury should resolve
that question as well. \textit{Chocktoot v.~Smith}, 571 P.2d 1255, 1259
(Or.~1977). But when the question is how a prior judge would have ruled on a legal
question, the authority is reversed, and courts tend to reserve those questions
for themselves. \textit{Tinelli v.~Redl}, 199 F.3d 603, 606--07 (2d Cir.~1999);
\textit{Phillips v.~Clancy}, 733 P.2d 300, 306 (Ariz.~Ct.~App.~1986);
\textit{Thomas v.~Bethea}, 718 A.2d 1187, 1197 n.7 (Md.~1998). The reasoning of
these cases, however, is open to disagreement. The observation that ``a judge
is best suited'' to answer a question, \textit{Royal Ins.~Co.~of
Am.~v.~Miles \& Stockbridge, P.C.},~133 F.~Supp.~2d 747, 762 (D.~Md.~2001), is
not usually a reason to withhold it from jurors, who, after all,
routinely hear difficult and complex cases. \textit{In re
U.S.~Fin.~Sec.~Litig.},~609 F.2d 411, 432 (9th Cir.~1979); \textit{see also}
\textit{Curtis v.~Loether}, 415 U.S.~189, 198 (1974) (risk of jury prejudice
``insufficient to overcome the clear command of the Seventh Amendment'');
\textit{Granfinanciera, S.A.~v.~Nordberg}, 492 U.S.~33, 63 (1989) (same for
risk of inefficiency); \textit{Hana Financial, Inc.~v.~Hana Bank}, 574
U.S.~418, 425 (2015) (same for risk that jury verdicts will be
``unpredictable''). And it is not clear that how a court \emph{would} have
acted in a counterfactual scenario is ``what the law is'' in the
\textit{Marbury} sense, \textit{see} \textit{Tinelli}, 199 F.3d at 607 (quoting
\textit{Marbury v.~Madison}, 5 U.S. 137, 177 (1803)), as it is purely
retrospective and cannot govern future behavior. It may be
that the tendency of courts to reserve for themselves cause-and-effect
questions about judges stems from uneasiness with the proposition that courts
are subject to ordinary cause-and-effect processes that juries can reason
about.

To assist in deciding how a prior litigation would have progressed absent the
lawyer's error, courts sometimes employ a device called the ``case within a
case'' (or ``suit within a suit''), in which the malpractice plaintiff
litigates the original case to a new judgment and the difference in outcomes is
taken to represent the plaintiff's damages. \textit{Witte v.~Desmarais}, 614
A.2d 116, 121 (N.H.~1992). The case-within-a-case collapses the distinction
between law \textit{about} the prior lawsuit and the law \textit{of} the prior
lawsuit by changing the causation question from what the result ``would have
been'' (law-about-law) to what it ``\textit{should have} been.'' \textit{See}
\textit{Harline v.~Barker}, 912 P.2d 433, 440 (Utah 1996) (emphasis in
original) (quoting 2 Ronald E.~Mallen \& Jeffrey M.~Smith, \textit{Legal
Malpractice} § 27.7, at 641). The former question is the usual but-for rule
from tort law, the latter a ``reflection or semblance'' of it when the former
question is difficult to answer. \textit{Lieberman v.~Employers Ins.~of
Wausau}, 419 A.2d 417, 427 (N.J.~1980).\footnote{There are other justifications
for the case-within-a-case approach, such as making the prior judge's testimony
irrelevant. \textit{Phillips}, 733 P.2d at 306.}

Mistaking the case-within-a-case approximation for the actual causation element
can lead to misleading results. For example, in arguing that settlement of an
underlying claim should bar a subsequent malpractice suit, the Pennsylvania Bar
Associations complains that allowing the malpractice suit would permit
``essentially a free second bite at the apple,'' a second roll of the dice in
which the original settlement acts as a safety net. \textit{Khalil
v.~Williams}, 53 EAL 2021 (Pa.~Aug.~3, 2021), Brief for Amici Curiae Pa.~Bar
Ass'n \textit{et al.}, at 11. But the Association's argument is only true under
the incorrect assumption that ``damages can only be measured against the result
the client would have obtained if the case had been tried to a final
judgment.'' \textit{Elizondo v.~Krist}, 415 S.W.3d 259, 263 (Tex.~2013);
\textit{Lieberman}, 419 A.2d at 427. In a malpractice case arising out of a
settlement, a court can avoid re-rolling the dice by requiring the client to
prove that ``absent malpractice, they probably would have recovered \emph{a
settlement} for more than'' the original amount. \textit{Elizondo}, 415 S.W.3d
at 270 (emphasis added). That proposition can be tested, for example, by
reference to ``settlements made under comparable circumstances.'' \textit{Id.}
Thus, while eliding the distinction between law \textit{about} the prior case
and the law \textit{of} the prior case can be useful for approximation
purposes, it is important to remember that the causation element of legal
malpractice is, ultimately, law-about-law.


\subsection{Liability Insurance}

A second example of law-about-law comes from liability insurance. ``The holder
of a liability insurance policy has a contractual right to payment \ldots when \ldots
the insured's liability to a third party is within the scope of the insurance
policy.'' 7A Couch on Ins.~§~103:11 (Westlaw 2022). When the insured has paid a
third-party claimant to satisfy an alleged liability, a question sometimes
arises whether the liability satisfied was within the scope of the insurance
policy. \textit{See, e.g.},~\textit{Liberty Mut.~Ins.~Co.~v.~Metro.~Life
Ins.~Co.},~260 F.3d 54, 58--59 (1st Cir.~2001). For example, the insured may
pay to settle a lawsuit asserting multiple claims, some covered and some not.
\textit{Crosby Estate at Rancho Santa Fe Master Ass'n v.~Ironshore Specialty
Ins.~Co.}, 578 F.~Supp.~3d 1123, 1134--35 (S.D. Cal. 2022). In such cases, the
coverage question turns on law-about-law: whether the legal basis for liability
fits within the policy coverage. \textit{See} \textit{id.}

It can occur that the law in effect at the time of payment differs from that in
effect at the time of indemnity. \textit{See} \textit{Cardinal v.~State}, 107
N.E.2d 569, 577 (N.Y.~1952). Even worse, the prior law may have been unsettled.
\textit{See} \textit{id.} The common-law system allows for gaps in authority.
Helena Whalen-Bridge, \textit{The Reluctant Comparativist:~Teaching Common Law
Reasoning to Civil Law Students and the Future of Comparative Legal Skills}, 58
J.~Legal Educ.~364, 367--68 (2008). A question arises whether the coverage
court should fulfill its ordinary common lawmaking function and fill the gaps.
It should not: in the law-\textit{about}-law, a conclusion that liability was unsettled
at the time of payment is sufficient to resolve the coverage question. \textit{See}
\textit{Cardinal}, 107 N.E.2d at 578 (``That question must be answered nunc pro
tunc, not by a long-later trial, under newly handed down decisions, as to who
could be held for damages, and on what theory~\ldots~.'').

Another issue that can arise is that an insured may justifiably settle a claim
even though it might have prevailed against the claimant at trial. \textit{See}
\textit{Luria Bros.~\& Co.~v.~Alliance Assurance Co.},~780 F.2d 1082, 1091 (2d
Cir.~1986). In that case, the insurer may argue that the payment falls outside
the policy because a \emph{liability} insurance policy requires ``actual,'' not
potential, liability. \textit{Catlin Specialty Ins.~Co.~v.~J.J.~White,
Inc.},~387 F.~Supp.~3d 583, 591 (E.D.~Pa.~2019) (Goldberg, J.). By analogy to
legal malpractice, it might seem that the insured must prove the
case-within-the-case in the coverage suit. \textit{Servidone
Const.~Corp.~v.~Security~Ins.~Co.~of Hartford}, 477 N.E.2d 441, 445
(N.Y.~1985). But courts have tended to reject that argument. \textit{Catlin},
387 F.~Supp.~3d at 590 (surveying cases). The insured need not prove that a
reasonable settlement of a covered claim was based on conduct that would have
resulted in actual liability to the underlying claimant. \textit{Id.};
\textit{see also} \textit{Chicago, R.I.~\& Pac.~Ry.~Co.~v.~United States}, 220
F.2d 939, 941 (7th Cir. 1955) (fact that party was not actually liable does not
mean it has no indemnity right for settlement). These decisions have tended to
rest on the policy ground of promoting settlement, \textit{e.g.},
\textit{Uniroyal, Inc.~v.~Home Ins.~Co.},~707 F.~Supp.~1368, 1378
(E.D.N.Y.~1988), but an argument from contract interpretation would be that a
contract insuring ``liability'' is law-about-law and thus makes the coverage
question turn on the claim against the insured rather than the insured's
conduct giving rise to the claim. To put it another way, a ``liability'' policy
is not a ``conduct'' policy.

The situation becomes more confusing if coverage turns on facts neither implied
by nor inconsistent with the insured's liability to the undermines claimant.
\textit{See} \textit{Catlin}, 387 F.~Supp.~3d at 594. For example, coverage may
depend on the date of conduct giving rise to injury, even though, since the
insured is not required to prove actual liability, the insured may be required
to prove the date of conduct that never occurred. \textit{Id.} As a
particularly perplexing example, after deciding that asbestos manufacturer
W.R.~Grace \& Co.~could obtain indemnity for settlements to alleged customers
for whom it did not actually install asbestos, one court observed that the
manufacturer was then seemingly required to prove ``the date of the
non-installation,'' a question about as comprehensible as the sound of one hand
clapping. \textit{Maryland Cas.~Co.~v.~W.R.~Grace \& Co.}, No.~88-cv-2613, 1996
WL 109068, at \*7 (S.D.N.Y.~Mar.~12, 1996). Courts have crafted various
solutions to this conundrum, but a general approach is not firmly established.
\textit{See} \textit{Catlin}, 387 F.~Supp.~3d at 594--96 (surveying
approaches).

Another puzzling situation occurs when the insurer attempts to use the
law-about-law coverage rule to its own advantage. In \textit{Apex Mortgage
Corp.~v.~Great Northern Insurance Co.},~972 F.3d 892 (7th Cir.~2020), an
insurer sought to disclaim coverage for an insured's settlement of a premises
liability claim based on an exclusion that would apply if the insured were ``in
possession'' of the property. \textit{Id.}~at 894. In arguing that the
insured's settlement established the applicability of the exclusion, the
insurer posited that liability under a premises theory could only attach if the
insured were in possession---and thus a claim under such a theory was
inherently outside the policy. \textit{Id.}~at 897-98. The Seventh Circuit
treated the issue as one of preclusion law and determined that the insurer's
argument failed because a settlement is not a ``judicial ruling.''
\textit{Id.}~at 898. But the issue is more accurately viewed as one of contract
law: whether the insurance contract was intended to cover risks related to
possession of the property. As an analogy, an insured could not avoid a fraud
exclusion in a professional negligence policy by arguing that a claim
denominated ``fraud'' was based on conduct that was \textit{in fact} only
negligent. \textit{Moscarillo v.~Professional Risk Management
Servs.,~Inc.},~921 A.2d 245, 256 (2007). Similarly, a commercial general
liability policy does not extend to claims of product defects even if the
damage were in fact inflicted by the insured's negligence. \textit{Kvaerner
Metals Div.~of Kvaerner U.S.,~Inc.~v.~Commercial Union Ins.~Co.},~908 A.2d 888,
896--97 (2006). To extend coverage based on the insured's actual conduct in
these cases would sweep in risks the policy was never meant to insure against.
\textit{See} \textit{id.}~at 899 (``To hold otherwise would be to convert a
policy for insurance into a performance bond.''). This observation is the
flip-side to the one stated above that refusing to indemnify a settlement of a
covered claim based on a lack of actual liability eliminates most of the value
of a ``liability'' policy. Although it may seem counterintuitive that actual
facts undermine the parties bargained-for coverage rather than clarify it,
understanding that a liability policy is law-about-law shows this to be the
correct approach.

In the malpractice context, it was noted above that courts sometimes elide the
distinction between law \textit{about} the prior litigation and the law
\textit{of} the prior litigation; doing so is administratively simple because
proving the case-within-a-case only requires the client to show what it has
maintained all along about the strength of its claim. \textit{See}
\textit{Lieberman}, 419 A.2d at 427. By contrast, asking an insured to prove a
case-within-a-case against itself would involve the insured ``turning about
face'' from contesting liability to embracing it. \textit{Uniroyal}, 707
F.~Supp.~at 1378. The awkwardness of that situation justifies only asking the
insured to demonstrate potential rather than actual liability. \textit{Id.};
\textit{see also} \textit{Lieberman}, 419 A.2d at 427 (suggesting
case-within-a-case not be used in malpractice action where there was a ``role
reversal''). Thus, recognizing that liability insurance is law-about-law is
necessary both to correctly enforce the scope of coverage and to avoid creating
a multitude of administrative problems in which settlement is all but
impossible. As will be shown below, these observations hold in other areas of
law-about-law as well.


\subsection{Reverse-Payment Patent Settlements}

A final example of law-about-law comes from an area that is not usually
considered to be ``about'' legal proceedings the way malpractice and liability
insurance are: antitrust. In \textit{FTC v.~Actavis, Inc.},~570 U.S.~136
(2013), the Supreme Court dealt with a so-called ``reverse-payment
settlement'': one in which a patentholder pays an alleged infringer not to
compete. In deciding whether such a settlement could give rise to a Sherman Act
violation, the \textit{Actavis} majority and dissent disagreed over whether a
competitor could violate the antitrust laws by paying to remove competition
that might have been precluded by a valid patent had the infringement suit
proceeded to judgment. \textit{See} \textit{id.}~at 164 (dissenting opinion).
In the dissent's view, if the patent were \emph{in fact} valid and infringed,
it could not be an antitrust violation to remove competition within its scope.
\textit{See} \textit{id.}~at 164, 171.

The majority correctly rejected this argument because it confuses law
\textit{about} the infringement litigation with the law \textit{of} the
infringement litigation. Within patent law, it is true that ``a patent is
either valid or invalid,'' no middle ground. \textit{Actavis}, 570 U.S.~at 171
(dissenting opinion). But when talking \textit{about} patent law, it is
accurate to acknowledge that a patent whose validity has not been adjudicated
``may or may not be valid.'' \textit{Id.}~at 147 (majority opinion). If a cloud
of uncertainty regarding the outcome of an infringement trial might spur the
patentholder to ``permit[] the patent challenger to enter the market before the
patent expires,'' it violates the antitrust laws to pay to remove that
competition. \textit{See} \textit{id.}~at 154.

There is an exceedingly close parallel between \textit{Actavis} and the
insurer--insured scenario discussed above. In both, the actor under scrutiny
negotiates its liability to one party while simultaneously owing a duty to
protect certain interests of third parties in the outcome of that negotiation.
In the insurer--insured context, the insured negotiates its liability to the
claimant while owing a duty to the insurer not to pay unreasonably or in bad
faith. \textit{Uniroyal}, 707 F.~Supp.~at 1379. In an \emph{Actavis} situation,
the alleged infringer negotiates its liability to the patentholder while owing
a duty to the consuming public not to unreasonably restrain trade. In both,
whether the negotiating party satisfies its duty is measured by its
\emph{potential} exposure rather than by litigating the case-within-a-case.
\textit{Luria}, 780 F.2d at 1091--92; \textit{Actavis}, 570 U.S.~at 157 (``[I]t
is normally not necessary to litigate patent validity to answer the antitrust
question~\ldots~.''). The following statements are therefore analogous:

{\vspace{5pt}\setstretch{1.0}%
\begin{enumerate}
  \item A liability insurer cannot avoid payment merely by showing that the
    insured would have prevailed against the underlying claimant.
    \textit{Luria}, 780 F.2d at 1091--92.
  \item An \textit{Actavis} defendant cannot avoid antitrust liability merely
    by showing that the patent would have been found valid and infringed had the 
    infringement suit proceeded to trial. \textit{See}
    \textit{Actavis}, 570 U.S.~at 147 (``[T]hat the agreement's
    `anticompetitive effects fall within the scope of the exclusionary
    potential of the patent' \ldots [does not] immunize the agreement from
    antitrust attack.'').
\end{enumerate}%
}\vspace{-12pt}%

As the \textit{Actavis} dissent observed, the FTC did not seek to ``assess the
validity of [the] patent[] or [the] question[] of infringement by bringing
[its] antitrust suit''---that is, the FTC did not seek to collaterally attack
the patent by showing it was \textit{in fact} not valid or not infringed.
\textit{Actavis}, 570 U.S.~at 164--65. The analogy to liability insurance shows
that the FTC was wise not to pursue such an attack: in defending such a suit,
the accused infringer would have to ``turn about face'' from contesting
infringement to embracing it, ``markedly reduc[ing] the advantages \ldots of
settling.'' \textit{Uniroyal}, 707 F.~Supp.~at 1378. For the same reason as in
liability insurance, playing out the case-within-a-case of patent
infringement would not only be theoretically incorrect: it would lead to a
quagmire in which settlement was all but impossible.

A final question is whether there is a textual justification for treating
antitrust law as law-about-law. Legal malpractice and liability insurance are
about prior legal proceedings and therefore inherently law-about-law, but the
same justification does not hold for antitrust, which is about ``trade.''
\textit{See} 15 U.S.C.~§~1. The \textit{Actavis} majority took a policy-based
approach, reasoning that the Sherman Act should take account of uncertainty in
patent law because Congress would have wanted to put antitrust policy on par
with patent policy in reconciling the two statutes. 570 U.S.~at 137. A possible
textual justification for doing so is that the word ``trade'' in the Sherman
Act includes trade in property rights and therefore refers also to the legal
proceedings that enforce those property rights. Thus, the Sherman Act is
textually law-about-law.


\section{Conclusion}

These are just three examples of law-about-law, but there are many more:
``clearly established'' law for purposes of 28 U.S.C. § 2254(d) or 42 U.S.C.
§ 1983, \textit{see} \textit{Pearson v.~Callahan}, 555 U.S.~223, 244 (2009),
``reasonable mistake[s] of law'' under the Fourth Amendment, \textit{Heien
v.~North Carolina}, 574 U.S.~54, 61 (2014), whether a statute is ambiguous and
an agency's interpretation of it reasonable, \textit{Chevron, U.S.A.,
Inc.~v.~NRDC}, 467 U.S.~837, 843 (1984), whether a seller misrepresented the
zoning status of a property, \textit{see} \textit{Lundin v.~Shimanski}, 368
N.W.2d 676, 679 (Wis.~1985), among others. Given the close parallels
illustrated between the three examples set out above---in particular the almost
perfect analogy between liability insurance and antitrust law---it is likely
that the same principles should govern other areas where the law talks about
law, and that these examples will provide guidance.


\end{document}

