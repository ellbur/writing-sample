% Options for packages loaded elsewhere
% vim: wrap
% vim: linebreak
% vim: spell
\documentclass[
  12pt,
  letterpaper,
]{scrartcl}

\usepackage[
  margin=1in,
  footskip=0.5in,
  footnotesep=14pt
]{geometry}

\usepackage{setspace}

\setlength{\parindent}{0.5in}
\setlength{\skip\footins}{10mm plus 2mm}
\setlength{\emergencystretch}{3em} % prevent overfull lines

\renewenvironment{quote}%
  {\list{}{\leftmargin=0.6in\rightmargin=0.6in}\item[]\setstretch{1.0}}%
  {\endlist\vspace{-\topsep}}

\usepackage[T1]{fontenc}
\usepackage[utf8]{inputenc}
\usepackage{textcomp}

\usepackage{calc} % for calculating minipage widths

\usepackage[xspace]{ellipsis}

\usepackage{soulutf8}
\setuldepth{o}
\renewcommand{\footnotesize}{\normalsize}

\addtokomafont{disposition}{\rmfamily}
\addtokomafont{section}{\normalsize\scshape}
\addtokomafont{subsection}{\normalsize}
\addtokomafont{subsubsection}{\normalsize}
\renewcommand{\thesection}{\Roman{section}} 
\renewcommand{\thesubsection}{\Alph{subsection}}
\renewcommand{\thesubsubsection}{\arabic{subsubsection}}

\renewcommand*{\sectionformat}{\thesection.\enskip}
\renewcommand*{\subsectionformat}{\makebox[0.5in]{}\thesubsection.\enskip}
\renewcommand*{\subsubsectionformat}{\makebox[1in]{}\thesubsubsection.\enskip}

\RedeclareSectionCommand[runin=false, afterindent=true,beforeskip=0.5\baselineskip,afterskip=0in]{section}
\RedeclareSectionCommand[runin=false, afterindent=true,beforeskip=0.25\baselineskip,afterskip=0in]{subsection}
\RedeclareSectionCommand[runin=false, afterindent=true,beforeskip=0.125\baselineskip,afterskip=0in]{subsubsection}

\title{Self-Conscious Common Law}
\author{Owen E. Healy}

\begin{document}

\begin{center}\vspace{1.0ex}{\noindent \large\textit{Self-Conscious Common Law}}\end{center}
  
\noindent\begin{minipage}{4in}
  \setlength{\parskip}{6pt}
  
  \noindent By: Owen E. Healy

  \noindent Written as an application writing sample. Not edited by any other person (or robot).
\end{minipage}

\setstretch{2.0}

\vspace{12pt}\section{Introduction}

How does the law answer questions about itself? That is---was law X
\emph{clearly established}? Did lawyer Y \emph{misrepresent} law X? Would court
Z have \emph{accepted} law X had the issue been raised? Statements about
statements (``higher-order statements'') have long bothered logicians,
\textit{see} W.V.~Quine, \textit{Philosophy of Logic} 66--68 (2d ed.~1986), so
it is not surprising that they would make trouble for judges as well. But
law-about-law is not just a theoretical curiosity. Untangling it matters to
judges who must decide whether to submit such ``legal facts'' to juries. And,
as shown below, the way law ``talks about law'' can affect the substance of
legal rights.

This essay surveys how courts deal with law-about-law in three discrete and
seemingly unrelated areas. The purpose is to demonstrate that all three are
variations of the same underlying issue---namely, how law talks about itself.


\section{Discussion}


\subsection{Legal Malpractice}

In a legal malpractice case arising out of litigation, the former client
accuses the lawyer of having litigated the case incorrectly. To prove
causation, the plaintiff must show that, had the lawyer acted correctly, the
outcome would have been more favorable. ???. The causation element is
law-about-law: how would another court have behaved differently had it been
presented with the correct litigation strategy? ???

This law-about-law question creates several issues. The first is whether a
judge or jury should decide it. ???. If the question is how a prior jury would
have acted on different evidence, there is general agreement to send that
question to a jury. ???. But when it is how a prior \emph{judge} would have
ruled on a different argument, the authority is reversed, and courts tend to
reserve those questions for themselves. ???. But the reasoning of cases so
holding is not convincing. It is doubtful that how a prior court would have
resolved a hypothetical dispute really falls on the ``law'' side of the law
versus fact dichotomy. It has none of the hallmarks of law: there are no
statutes addressing it, ??? (???search for source defining ``law''???) The
argument that judges are better than juries at predicting how other judges
would act, even if true, is not a reason for withholding a factual question
from the jury. ???. Juries routinely resolve difficult factual questions. ???.
The Seventh Amendment makes no exception for questions about which judges think
they can do a better job. ???. Judges should not be shy of the fact that they,
too, are subject to the laws of cause and effect and juries can reason about
their actions.

To assist in deciding how a prior litigation would have progressed had the
lawyer taken a different approach, courts sometimes employee a device called the
``case within a case,'' in which the malpractice plaintiff litigates the
original case to a new judgment and the difference in outcomes is taken to
represent the plaintiff's damages. ???. But it is important to understand that
the case-within-a-case is not actually how the prior litigation would have
resolved---it is an approximation device. Unfortunately, failure to appreciate
this distinction has led to some bad law. ??? Pennsylvania example ???


\subsection{Insurer--Insured}

A liability insurance policy gives the insured a right to indemnity for
payments made to satisfy certain kinds of liability of the insured to third
parties. ???. After the insured has paid, a question sometimes arises whether
the liability was covered. ???. For example, the insured may pay to settle a
lawsuit asserting multiple claims, some covered and some not. ???. In such
cases, the coverage question turns on law-about-law: whether the legal basis
for liability fits within the legal test for coverage.

It may be that the law in effect at the time of payment differs from that in
effect at the time of indemnity. ???. Even worse, the prior law may have been
unsettled. ???. The common-law system allows for gaps in authority---in fact,
those `gaps \emph{are} the law' because they represent the extension of old
authority to new facts. ???. A question arises whether the coverage court
should fulfill its ordinary common lawmaking function and fill the gaps. It
should not: in the law-about-law, a conclusion that liability was unsettled at
the time of payment is sufficient to resolve coverage. ???.

Another issue that may arise is that an insured may justifiably settle a claim
even though it could have prevailed against the claimant at trial. ???. In that
case, the insurer will argue that the payment falls outside the policy
because a \emph{liability insurance} policy requires \emph{liability}. ???. By
analogy to legal malpractice, it might seem that the insured must prove the
case-within-the-case in the coverage suit. But courts have tended to reject
that argument. ???. The key is to observe that coverage is a question of
law-about-law and thus depends on the characterization of the \emph{claim}, not
of the insured's underlying conduct that gave rise to the claim---it is
a liability policy, not a conduct policy. Thus, the insured need not
prove that a reasonable settlement of a covered claim was based on actual
liability to the underlying claimant. ???.

The confusion becomes worse if coverage turns on facts neither established nor
refuted by the insured's liability. ???. For
example, coverage may depend on the date of conduct giving rise to injury. ???.
But, since the insured is not required to prove actual liability, the insured
may be required to prove the date of conduct that might never have occurred at
all. ???. As a particularly perplexing example, after deciding that asbestos
manufacturer W.R.~Grace \& Co.~could obtain indemnity for settlements to
alleged customers for whom Grace did not actually install asbestos, the court
observed that, to satisfy the coverage period, Grace
was seemingly required to prove ``the date of noninstallation,'' a question
about as comprehensible as the sound of one hand clapping. ???. As another
example, a manufacturer seeking coverage for liability to purchasers of
silicone breast implants was required to prove the \emph{date} of injury (but
not the \emph{fact} of injury), even though science conclusively established
that the product caused no injury at all. ???; \textit{see also}
\textit{Catlin}.

It may be tempting to view the effect of settlement on a later indemnity suit
through the lens of preclusion doctrine. ???. In ???, for example, it has
become common practice for plaintiffs settling with an insured to obtain a pro
forma default judgment establishing the insured's liability in a subsequent
coverage action. ???. But that commits the same error as assuming that the
insured must prove actual liability---that is, it confuses law-about-law
(whether the claim is covered) with law-about-facts (whether the conduct is
covered). Thus, whether a policy affords coverage to a prior settlement is a
matter of contract, not preclusion, law. ???.

Although there may be little a court can do to make these bizarre
counterfactuals palatable to juries, at least by acknowledging that coverage
under a liability policy is a question of law-about-law, a court can satisfy
itself that the fault is not that the tests are wrong but that statements about
statements are inherently confusing.


\subsection{\textit{FTC v.~Actavis}}

In \textit{FTC v.~Actavis ???}, ???, the Supreme Court dealt with the antitrust
implications of so called ``reverse-payment settlements'' in patent
infringement suits: settlements in which the patentholder pays the alleged
infringer not to compete. In deciding whether such settlements could give rise
to a Sherman Act claim, a primary point of disagreement between the
\textit{Actavis} majority and the dissent was whether a competitor could
violate the antitrust laws by paying to remove competition that might have been
precluded by a valid patent had the infringement suit proceeded to judgment.
???. In the dissent's view, if the patent were \emph{in fact} valid and
infringed, it could not be an antitrust violation to remove the potential
competition created by the cloud of uncertainty as to whether the accused
infringer might prevail at trial. ???. The dissent also argued that treating
potential competition as actual competition protected by the antitrust laws
was contrary to the more-likely-than-not test for preponderance of the evidence
in civil cases. ???.

The error in the dissent's reasoning, and the reason the majority was right to
reject it, is that it confuses law-about-law with law-about-fact. If a cloud of
uncertainty regarding the outcome of a potential infringement trial would have
spurred the patentholder to allow the alleged infringer on the market, that is
actual competition, not just potential competition. And it does not violate the
more-likely-than-not test for a jury to consider whether it is more likely than
not that the patentholder paid to remove that cloud of uncertainty.

There is a strangely close parallel between \textit{Actavis} and the
insurer--insured scenario discussed above. In both, the actor under scrutiny
negotiates its liability to one party while simultaneously holding a duty to
protect certain interests of a third party in the outcome of that negotiation.
In the insurer--insured context, the insured negotiates its liability to the
claimant while owing a duty to the insurer not to pay in bad faith. ???. In an
\emph{Actavis} situation, the alleged infringer negotiates its liability to the
patentholder while owing a duty to the consuming public not to unreasonably
restrain trade. ???. In both, whether the negotiating party satisfies that duty
is measured by its \emph{potential} exposure rather than by litigating the
case-within-a-case and seeing whether liability would have been found at trial.
That is, just as the insurer cannot avoid payment by showing that the insured
would have prevailed against the underlying claimant, an antitrust plaintiff
cannot show that an alleged infringer acted wrongly merely by settling a claim
it might have won. ???.

Tellingly, the \textit{Actavis} dissent was not willing to let the FTC prove an
antitrust violation by relitigating the infringement case and showing that the
alleged infringer would have prevailed. ???. And with good reason: infringement
suits would never settle if doing so would not obtain finality. Courts have
recognized exactly the reason for not allowing a liability insurer to avoid
coverage by relitigating the underlying claim. ???.


\section{Conclusion}


\end{document}
